\documentclass{spral}

\newcommand{\packagename}{Installation}
\newcommand{\version}{1.0.0}
\newcommand{\versiondate}{6 March 2014}
\newcommand{\purpose}{
The Sparse Parallel Robust Algorithm Library (SPRAL) is an open-source library
for sparse linear algebra and related algorithms. It is primarily developed and
maintained by the Numerical Analysis Group at STFC's Rutherford Appleton Laboratory.
}

\begin{document}

\title{SPRAL \\Installation Instructions}
\author{Jonathan Hogg (STFC Rutherford Appleton Laboratory)}
% We don't use \maketitle as these instructions are somewhat special
\makeatletter
\pgfimage[height=1cm]{stfc}
\hfill
{\Huge \bfseries \textcolor{stfcblue}{\libraryname}}
\vspace{0.1cm}
\textcolor{stfcgrey}{\hrule}
\vspace{0.5cm}

\begin{center}
   \LARGE \bfseries
   \@title
\end{center}
\begin{quote}
   \large
   \purpose
\end{quote}

\begin{flushright}
\noindent
\@author
\end{flushright}
\vspace{-0.2cm}
\textcolor{stfcgrey}{\hrule}
\makeatother
\thispagestyle{firststyle}

\versionhistory
\begin{description}
\item[2014-03-19 Version 1.0.0] Initial release
\end{description}

%%%%%%%%%%%%%%%%%%%%%% HOW TO USE %%%%%%%%%%%%%%%%%%%%%%%%

\section{Quick Start}

Under Linux, or Mac OS X:

\begin{verbatim}
# Build and install library
BUILDDIR=build; mkdir $BUILDDIR; cd $BUILDDIR
../configure --with-metis="-L/path/to/metis -lmetis"
make
sudo make install # Optional

# Link against library
cd /path/to/your/code
gfortran -o myprog myobj.o -lspral -lmetis -lblas
\end{verbatim}

\subsection*{Notes}
\begin{itemize}
   \item The above instructions perform an out-of-path build: we recommend you
      change the value of \texttt{\$BUILDDIR} to reflect the compiler used, for
      example \texttt{BUILDDIR=build-gfortran-4.8}.
   \item The above will build the SPRAL library and a number of driver programs.
      To build the example and test codes, use \texttt{make check}.
   \item Installation is not required: in many cases it will be sufficient to
      just link against the static library found in the \texttt{.libs}
      subdirectory.
   \item If you write a paper using software from SPRAL, please cite an
      appropriate paper (a list can usually be found in the method section of
      the user documentation). If none is listed, a citation of the library
      website should be used:
      \begin{quote}
         SPRAL: an open-source library for sparse linear algebra, Version 2014-03-20, \url{http://www.numerical.rl.ac.uk/spral}, March 2014.
      \end{quote}
\end{itemize}

\section{Licence}
At present, all code is covered by the 3-clause BSD licence given in the file
LICENCE.

\section{Third-party libraries}
\subsection{METIS}
Many of our packages use the METIS graph partitioning library to find good
orderings. You can obtain a copy of METIS from the following url:\\
\url{http://www.cs.umn.edu/~metis}\\
We support both version 4 and version 5 (the latter is available under the open
source Apache Licence). At present, we recommend version 4 as it is faster than
version 5, though this may change in the future.\\
If the METIS library is not available on the default link path, the
\texttt{--with-metis} option to \texttt{configure} should be used to specify
how to link against METIS. For example, if \texttt{libmetis.a} is in the directory \texttt{/usr/local/metis-4.0}, use: \\
\texttt{../configure --with-metis="-L/usr/local/metis-4.0 -lmetis"}

\subsection{BLAS}
Many of our packages require a high performance BLAS library to efficiently
perform dense linear algebra operations. For best performance, please use the
library recommended by your computer manufacturer (normally the Intel MKL).
If this is not available, use an optimized alternative, such as OpenBLAS.
The reference BLAS from netlib are at least an order of magnitude slower than
modern optimized BLAS, and should be avoided. If bit-compatible results are
desired, a bit-compatible BLAS library must be used.

If the BLAS library is not available on the default link path, or if
\texttt{configure} detects the wrong BLAS library, the \texttt{--with-blas}
option to \texttt{configure} should be used to specify how to link against
the BLAS library. For example, to link against the Intel MKL using the GNU
compiler suite, use: \\
\begin{verbatim}
../configure --with-blas="-L/opt/intel/mkl/lib/intel64 -lmkl_gf_lp64 -lmkl_gnu_thread \
   -lmkl\_core"
\end{verbatim}

\section{Compilers and compiler options}
If no compiler is specified, \texttt{configure} will pick a default
compiler to use. If \texttt{configure} cannot find an appropriate compiler, or
you wish to specify a different compiler you can do so by setting the following
variables:
\begin{description}
   \item[CC] specifies the C compiler to use.
   \item[FC] specifies the Fortran 90/95/2003/2008 compiler to use.
   \item[NVCC] specifies the CUDA compiler to use.
\end{description}
\vspace{0.1cm}

\noindent
Additionally, compiler flags can be specified using the following variables:
\begin{description}
   \item[CFLAGS] specifies options passed to the C compiler.
   \item[FCFLAGS] specifies options passed to the Fortran compiler
   \item[NVCCFLAGS] specifies options passed to the CUDA compiler.
\end{description}
\vspace{0.1cm}

\noindent
For example, to compile with \texttt{ifort -g -O3 -ip} we could use:
\begin{verbatim}
../configure FC=ifort FCFLAGS="-g -O3 -ip"
\end{verbatim}

\section{Other options to \texttt{configure}}
The \texttt{configure} script is generated by \texttt{autoconf}, and hence
offers many standard options that can be listed using the command
\texttt{configure --help}.

\noindent
In particular, the following options may be of interest:
\begin{description}
   \item[\texttt{--prefix=PREFIX}] specifies the installation path prefix (by default \texttt{/usr/local/} under Linux).
\end{description}

\section{Support}
Feeback may be sent to hsl@stfc.ac.uk.\\
Bugs can be reported online via our bug tracking system: \\
\url{http://ccpforge.cse.rl.ac.uk/gf/project/spral/tracker/?action=TrackerItemBrowse&tracker_id=498}. \\
We will endeavour to fix all reported bugs.

\end{document}
